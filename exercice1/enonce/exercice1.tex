\documentclass{exam}
    \usepackage{matlab-prettifier}
    \usepackage[twocolumn]{geometry}
    \usepackage[small]{titlesec}
    \usepackage{enumitem}
    \usepackage[colorlinks=true,urlcolor=blue]{hyperref}
    \setlist[description]{nosep,labelindent=1em}
    
    \lstset{style=Matlab-editor,aboveskip=0.6pt,belowskip=-0.4pt}
    
    %Reduce spacing for subsection
    \titlespacing\subsection{0pt}{-1pt}{-1pt}
    %Header config
    \pagestyle{head}
    \extrafootheight{-.75in}
    \header{INF-135}{Principes fondamentaux, structures de contrôles conditionnelles}{Exercices 1}
    
    \qformat{\textbf{Exercice \thequestion}\quad \thequestiontitle\hfill}
    \begin{document}
    \begin{questions}
        \titledquestion{Allo monde}
        \textbf{Nom du script}
E1AlloMonde.m\\
\textbf{Description}
Vide la fenêtre de commande, affiche le message "Allo monde!" avec un saut de ligne dans la fenêtre de commande.
        \titledquestion{Horaire}
        \textbf{Nom du script}
E2Horaire.m\\
\textbf{Description}
Affiche l'horaire à l'écran : \\ \begin{tabular}{ | l | } \hline Horaire:          \\ 7h00 – déjeuner \\ 12h00 – diner    \\ 17h00 – souper   \\ \hline \end{tabular}          
        \titledquestion{Nombre double}
        \textbf{Nom du script}
E3DoubleDuNombre.m\\
\textbf{Description}
Demande à l’utilisateur de saisir un nombre, le script affiche ensuite le double (multiplication par 2) du nombre. L'affichage prends la forme "Double du nombre : NOMBRE" suivi d'un saut de ligne.
        \titledquestion{Nombre carré}
        \textbf{Nom du script}
E4CarreDuNombre.m\\
\textbf{Description}
Demande à l'utilisateur de saisir un nombre, le script affiche ensuite le nombre au carré (multiplié par lui-même). L'affichage prends la forme "Carré du nombre : NOMBRE" suivi d'un saut de ligne.
        \titledquestion{Coûts d’achats avec taxes}
        \textbf{Nom du script}
E5CoutAvecTaxes.m\\
\textbf{Description}
Calcule le prix à payer sur un article achetés plusieurs foix incluant les taxes. Le script saisit le nombre d'articles achetés et le prix de l'article acheté. La script affiche ensuite le prix total en ajoutant une taxe de vente de 13\%. L'affichage prends la forme "Prix total : PRIX" suivi d'un saut de ligne. L'affichage du prix inclus seulement deux décimales.
        \titledquestion{Nom et prénom}
        \textbf{Nom du script}
E6NomEtPrenom.m\\
\textbf{Description}
Demande à l’utilisateur de saisir le nom suivi du prénom de l'utilisateur (avec deux saisis disctinctes). Le script affiche les message suivant avec le nom et prénom de l'utilisateur "Bonjour PRENOM NOM!" suivi d'un saut de ligne.
        \titledquestion{Multiplication de deux nombres}
        \textbf{Nom du script}
E7MultiplieDeuxNombres.m\\
\textbf{Description}
Demande à l'utilisateur de saisir deux nombres et affiche le produit des deux nombres. L'affichage prends la forme "NOMBRE1 x NOMBRE2 = PRODUIT" suivi d'un saut de ligne.
        \titledquestion{Positif, négatif ou nul}
        \textbf{Nom du script}
E8PositifNegatifNul.m\\
\textbf{Description}
Demande à l'utilisateur de saisir un nombre. Le script affiche si le nombre est positif, négatif ou nul. L'affichage prends la forme "Le nombre est positif.", "Le nombre est négatif." ou "Le nombre est nul." tous suivis d'un saut à la ligne.
        \titledquestion{Minimum}
        \textbf{Nom du script}
E9MinimumEntreDeux.m\\
\textbf{Description}
Demande à l'utilisateur de saisir deux nombres. Le script affiche ensuite le plus petit des deux nombres. L'affichage prends la forme "Plus petit : NOMBRE" suivi d'un saut de ligne.
        \titledquestion{Quadrant cartésien}
        \textbf{Nom du script}
E10QuadrantCartesien.m\\
\textbf{Description}
Demande à l'utilisateur de saisir deux coordonnées. Les saisies représente les coordonnées x et y d'un point sur un plan cartésien. Le script affiche dans quel quadrant appartient le point. L'affichage prends la forme "(X, Y) est dans le quadrant : QUADRANT." suivi d'un saut de ligne. La valeur QUADRANT peut être "I", "II", "III" ou "IV". \href{https://fr.wikipedia.org/wiki/Quadrant_(mathématiques)}{Quadrant Cartésiens}
        \titledquestion{IMC}
        \textbf{Nom du script}
E11IMC.m\\
\textbf{Description}
Demande à l'utilisateur de saisir son poids (en kg) et sa taille (en m). Le script calcule l'indice de masse corporelle en utilisant la forumule suivante: $$IMC = \frac{poids}{taille^2}$$ Le script affiche ensuite la catégorie de l'IMC selon les critères suivant : si l'IMC est plus petit que 18.5, l'utilisateur est "maigre", si l'IMC est plus grand ou égal à 18.5 et plus petit que 30, l'utilisateur est "normale" et si l'utilisateur est plug grand ou égal à 30, l'utilisateur est "obèse". L'affichage prends la forme "Catégorie IMC : CATÉGORIE" suivi d'un saut de ligne. \href{https://fr.wikipedia.org/wiki/Indice_de_masse_corporelle}{IMC}
        \titledquestion{Plus petit de trois}
        \textbf{Nom du script}
E12MinimumEntreTrois.m\\
\textbf{Description}
Demande à l'utilisateur de saisir trois nombres. Le script affiche le plus petit des trois nombres. Le script doit aussi affiché un message si il y a eu égalité sur le plut petit nombre. L'affichage de base prends la forme "Plus petit : NOMBRE" suivi d'un saut de ligne. L'affichage de l'égalité prends la forme "Il y a eu égalité." suivi d'un saut de ligne.
        \titledquestion{Menu}
        \textbf{Nom du script}
E13Menu.m\\
\textbf{Description}
Affiche un menu qui demande à l'utilisateur de saisir un nombre qui représente le numéro de l'exercice du programme que celui-ci souhaite exécuter parmi les différents exercices réalisés jusqu'à présent. L'affichage du menu prends la forme suivante. La première ligne sera "Quel programme souhaitez-vous appeler?" suivi d'un saut de ligne. Ensuite, chaque ligne aura la forme de "NUMERO - NOMSCRIPT" suivi d'un saut de ligne. Si la valeur entrée est valide, le script appelle le programme demandé. Sinon, le script affiche un message d'erreur indiquant que la sélection est invalide. L'affichage du message d'erreur prends la forme "Choix invalide." suivi d'un saut de ligne.
        
    \end{questions}
\end{document}