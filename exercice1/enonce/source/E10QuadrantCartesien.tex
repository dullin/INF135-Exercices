\textbf{Nom du script}
E10QuadrantCartesien.m\\
\textbf{Description}
Demande à l'utilisateur de saisir deux coordonnées. Les saisies représente les coordonnées x et y d'un point sur un plan cartésien. Le script affiche dans quel quadrant appartient le point. L'affichage prends la forme "(X, Y) est dans le quadrant : QUADRANT." suivi d'un saut de ligne. La valeur QUADRANT peut être "I", "II", "III" ou "IV". \href{https://fr.wikipedia.org/wiki/Quadrant_(mathématiques)}{Quadrant Cartésiens}